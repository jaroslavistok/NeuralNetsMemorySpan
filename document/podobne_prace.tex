\chapter{Práce zaoberajúce sa hĺbkou pamäti neurónových sietí}

Do podobných prác sme vybrali vedecké články, ktoré sa zaoberajú rekurentnými neurónovými sieťami aj rekurentnými 
samoorganizujúcimi sa mapami.
Nasledujúce dve práce sa zaoberajú vlastnosťami SRN sietí:
\begin{itemize}
    \item \textbf{
        Graded State Machines: The Representation of Temporal Contingencies in Simple Recurrent Networks \cite{Servan-Schreiber1991}}
         \\ \\
    Autori tejto práce analyzovali reprezentacie, 
    ktore sa vyvinuli v skrytom stavovom priestore SRN a zistili, 
    že výstup siete v priebehu trénovania závisí od čoraz dlhších 
    historických kontextov. 
    Zistili tiež, že tento typ neurónovej siete si dokáže 
    vo svojom stavovovom priestore vytvoriť stavový automat alebo aj počítadlo.

    \item \textbf{
        Predicting the Future of Discrete Sequences from Fractal Representations of the Past  \cite{Tino2001}} \\ \\
    Autori tejto práce zistili, že SRN funguje podobne ako markovoský model s variabilnou dĺžkou (VLMM). 
    To znamená, že sieť sa naučí kontexty dĺžky potrebnej pre správny výstup. 
    Navrhli tiež tzv. fractal prediction machine, ktorou sa inšpirovali v práci Markovian architectural bias of recurrent neural networks \cite{markovnian_bias} ktorí vytvorili neural prediction machine z nantrénovanej SRN.
    Neural prediction machines využívajú fraktálovú vlastnosť SRN, to znamená, 
    že postupnosti so spoločným postfixom zobrazujú blízko seba v stavovom priestore siete. 
    Čo je zaujímavé je, že takúto vlastnosť má dokonca aj nenatrénovaná sieť, čo sa nazýva architectural bias.
\end{itemize}

Doteraz sme sa zaoberali prácami, ktoré sa zaoberali vlastnosťami SRN.
Ako poslednú uvádzame prácu, ktorá sa zaoberá RecSOM a jej vlastnosťami:
\begin{itemize}
    \item \textbf{Recursive Self-organizing Map
    as a Contractive Iterative Function System \cite{rsm}}
    Autori tejto práce analyzovali RecSOM a zistili, že za určitých 
    špecifických podmienok má, podobne ako SRN, takisto fraktálovú organizáciu. 
    Odvodili vzorec pre maximálnu hodnotu alpha parametra (rovnica \ref{eq:rec_som_distance}), 
    pri ktorej je táto vlastnosť garantovaná.
\end{itemize}


