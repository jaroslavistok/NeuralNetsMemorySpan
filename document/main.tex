\documentclass[12pt, oneside]{book}
\usepackage[a4paper,top=2.5cm,bottom=2.5cm,left=3.5cm,right=2cm,footskip=.25in]{geometry}
\usepackage[utf8]{inputenc}
\usepackage[T1]{fontenc}
\usepackage{graphicx}
\usepackage{url}
\usepackage{float}
\usepackage{mathtools}
\usepackage{amsmath}
\usepackage{amssymb}
\usepackage{lmodern}
\usepackage{indentfirst}
\usepackage{enumerate}
\usepackage{array}
\usepackage{ragged2e}
\usepackage{lipsum}

\usepackage[
backend=bibtex
% backend=biber % <-- biber is a modern replacement for BibTeX designed for use (only) with biblatex
]
{biblatex}
\addbibresource{bibliography.bib}

\usepackage{hyperref}
\hypersetup{
    colorlinks,
    citecolor=black,
    filecolor=black,
    linkcolor=black,
    urlcolor=black
}

\usepackage[slovak]{babel}
%\renewcommand{\baselinestretch}{1.8} 

% toto som zmenil aby som mal viac textu na seminar :D
\linespread{1.25} % hodnota 1.25 by mala zodpovedat 1.5 riadkovaniu

% -------------------
% --- Definicia zakladnych pojmov
% --- Vyplnte podla vasho zadania
% -------------------
\def\mfrok{2019}
\def\mfnazov{Porovnanie niekoľkých typov rekurentných sietí z hľadiska hĺbky pamäte}
\def\mftyp{Diplomová práca}
\def\mfautor{Jaroslav Ištok}
\def\mfskolitel{školitel}


%ak mate konzultanta, odkomentujte aj jeho meno na titulnom liste
%\def\mfkonzultant{ }  

\def\mfmiesto{Bratislava, \mfrok}

%aj cislo odboru je povinne a je podla studijneho odboru autora prace
\def\mfodbor{2511  Aplikovaná informatika} 
\def\program{ Aplikovaná informatika }
\def\mfpracovisko{ Katedra aplikovanej informatiky }

\begin{document}     

% -------------------
% --- Obalka ------
% -------------------
\thispagestyle{empty}

\begin{center}
\sc\large
 Univerzita Komenského v Bratislave\\
 Fakulta matematiky, fyziky a informatiky
\vfill

{\LARGE\mfnazov}\\
\mftyp

\end{center}

\vfill

{\sc\large 
\noindent \mfrok\\
\mfautor
}

\eject % EOP i
% --- koniec obalky ----

% -------------------
% --- Titulný list
% -------------------

\thispagestyle{empty}
\noindent
"" 
\begin{center}
\sc  
\large
 Univerzita Komenského v Bratislave\\
 Fakulta matematiky, fyziky a informatiky

\vfill

{\LARGE\mfnazov}\\
\mftyp
\end{center}

\vfill

\noindent
\begin{tabular}{ll}
Študijný program: & \program \\
Študijný odbor: & \mfodbor \\
Školiace pracovisko: & \mfpracovisko \\
Školiteľ: & \mfskolitel \\
% Konzultant: & \mfkonzultant \\
\end{tabular}


\vfill


\noindent \mfmiesto\\
\mfautor

\eject % EOP i


% --- Koniec titulnej strany


% -------------------
% --- Zadanie z AIS
% -------------------
% v tlačenej verzii s podpismi zainteresovaných osôb.
% v elektronickej verzii sa zverejňuje zadanie bez podpisov

\newpage 
\thispagestyle{empty}
\hspace{-2cm}\includegraphics[width=1.1\textwidth]{assets/zadanie_master_thesis}


% --- Koniec zadania

\frontmatter

% -------------------
%   Čestné výhlásenie 
% -------------------
\setcounter{page}{3}
\newpage 
~
\vfill
{\bf Čestné vyhlásenie }

% --- Koniec čestného  výhlásenia



% -------------------
%   Poďakovanie - nepovinné
% -------------------
\setcounter{page}{3}
\newpage 

~
\vfill
{\bf Poďakovanie }

% --- Koniec poďakovania


% -------------------
%   Abstrakt - Slovensky
% -------------------
\newpage 
\section*{Abstrakt}

Abstrakt - obsah
 

\paragraph*{Kľúčové slová:} rekurentné neurónové siete, hĺbka pamäte
% --- Koniec Abstrakt - Slovensky


% -------------------
% --- Abstrakt - Anglicky 
% -------------------
\newpage 
\section*{Abstract}

Abstract

\paragraph*{Keywords:} neural net, machine learning, memory span

% --- Koniec Abstrakt - Anglicky

% -------------------
% --- Predhovor - v informatike sa zvacsa nepouziva
% -------------------
%\newpage 
%\thispagestyle{empty}
%
%\huge{Predhovor}
%\normalsize
%\newline
%Predhovor je všeobecná informácia o práci, obsahuje hlavnú charakteristiku práce 
%a okolnosti jej vzniku. Autor zdôvodní výber témy, stručne informuje o cieľoch 
%a význame práce, spomenie domáci a zahraničný kontext, komu je práca určená, 
%použité metódy, stav poznania; autor stručne charakterizuje svoj prístup a svoje 
%hľadisko. 
%
% --- Koniec Predhovor


% -------------------
% --- Obsah
% -------------------

\newpage 

\tableofcontents

% ---  Koniec Obsahu

% -------------------
% --- Zoznamy tabuliek, obrázkov - nepovinne
% -------------------

\newpage 

\listoffigures

% ---  Koniec Zoznamov

\mainmatter


\input uvod.tex 
\input navrh_riesenia.tex
\input implementacia.tex
\input experiment.tex
\input vyhodnotenie.tex

\input zaver.tex

% -------------------
% --- Bibliografia
% -------------------


 \addcontentsline{toc}{chapter}{Bibliography}
\newpage	

\backmatter

\thispagestyle{empty}
\nocite{*}
\clearpage

\printbibliography

% -------------------
%--- Prilohy---
% -------------------

%Nepovinná časť prílohy obsahuje materiály, ktoré neboli zaradené priamo  do textu. Každá príloha sa začína na novej strane.
%Zoznam príloh je súčasťou obsahu.
%
%\addcontentsline{toc}{chapter}{Appendix A}
%\input AppendixA.tex
%
%\addcontentsline{toc}{chapter}{Appendix B}
%\input AppendixB.tex

\end{document}






