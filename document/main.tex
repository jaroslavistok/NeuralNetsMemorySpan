\documentclass[12pt, oneside]{book}
\usepackage[a4paper,top=2.5cm,bottom=2.5cm,left=3.5cm,right=2cm]{geometry}
\usepackage[utf8]{inputenc}
\usepackage[T1]{fontenc}
\usepackage{graphicx}
\usepackage{url}


\usepackage{float}
\usepackage{mathtools}
\usepackage[slovak]{babel} % vypnite pre prace v anglictine
\linespread{1.25} % hodnota 1.25 by mala zodpovedat 1.5 riadkovaniu

% -------------------
% --- Definicia zakladnych pojmov
% --- Vyplnte podla vasho zadania
% -------------------
\def\mfrok{2019}
\def\mfnazov{Memory Span In Different Recurrent Neural Network Types}
\def\mftyp{Master Thesis}
\def\mfautor{Jaroslav Ištok}
\def\mfskolitel{školitel}

%ak mate konzultanta, odkomentujte aj jeho meno na titulnom liste
%\def\mfkonzultant{ }  

\def\mfmiesto{Bratislava, \mfrok}

%aj cislo odboru je povinne a je podla studijneho odboru autora prace
\def\mfodbor{2511  Applied Informatics} 
\def\program{ Applied Informatics }
\def\mfpracovisko{ Department of Applied Informatics }

\begin{document}     

% -------------------
% --- Obalka ------
% -------------------
\thispagestyle{empty}

\begin{center}
\sc\large
 Comenius University in Bratislava\\
 Faculty of Mathematics, Physics and Informatics

\vfill

{
\LARGE\mfnazov}\\
\mftyp

\end{center}

\vfill

{\sc\large 
\noindent \mfrok\\
\mfautor
}

\eject % EOP i
% --- koniec obalky ----

% -------------------
% --- Titulný list
% -------------------

\thispagestyle{empty}
\noindent
""
\begin{center}
\sc  
\large
 Comenius University in Bratislava\\
 Faculty of Mathematics, Physics and Informatics

\vfill

{\LARGE\mfnazov}\\
\mftyp
\end{center}

\vfill

\noindent
\begin{tabular}{ll}
Study programmes : & \program \\
Branch of study: & \mfodbor \\
Educational department: & \mfpracovisko \\
Advisor: & \mfskolitel \\
% Konzultant: & \mfkonzultant \\
\end{tabular}

\vfill


\noindent \mfmiesto\\
\mfautor

\eject % EOP i


% --- Koniec titulnej strany


% -------------------
% --- Zadanie z AIS
% -------------------
% v tlačenej verzii s podpismi zainteresovaných osôb.
% v elektronickej verzii sa zverejňuje zadanie bez podpisov

\newpage 
\thispagestyle{empty}
\hspace{-2cm}\includegraphics[width=1.1\textwidth]{assets/zadanie_master_thesis}


% --- Koniec zadania

\frontmatter

% -------------------
%   Poďakovanie - nepovinné
% -------------------
\setcounter{page}{3}
\newpage 
~

\vfill
{\bf Acknowledgment: }

% --- Koniec poďakovania

% -------------------
%   Abstrakt - Slovensky
% -------------------
\newpage 
\section*{Abstrakt}

Abstrakt - obsah
 

\paragraph*{Kľúčové slová:} kľúčové slová
% --- Koniec Abstrakt - Slovensky


% -------------------
% --- Abstrakt - Anglicky 
% -------------------
\newpage 
\section*{Abstract}

Abstract


\paragraph*{Keywords:} Neural Net, Machine Learning, Prediction

% --- Koniec Abstrakt - Anglicky

% -------------------
% --- Predhovor - v informatike sa zvacsa nepouziva
% -------------------
%\newpage 
%\thispagestyle{empty}
%
%\huge{Predhovor}
%\normalsize
%\newline
%Predhovor je všeobecná informácia o práci, obsahuje hlavnú charakteristiku práce 
%a okolnosti jej vzniku. Autor zdôvodní výber témy, stručne informuje o cieľoch 
%a význame práce, spomenie domáci a zahraničný kontext, komu je práca určená, 
%použité metódy, stav poznania; autor stručne charakterizuje svoj prístup a svoje 
%hľadisko. 
%
% --- Koniec Predhovor


% -------------------
% --- Obsah
% -------------------

\newpage 

\tableofcontents

% ---  Koniec Obsahu

% -------------------
% --- Zoznamy tabuliek, obrázkov - nepovinne
% -------------------

\newpage 

\listoffigures

% ---  Koniec Zoznamov

\mainmatter


\input uvod.tex 

\input kapitola.tex

\input zaver.tex

% -------------------
% --- Bibliografia
% -------------------

 \addcontentsline{toc}{chapter}{Bibliography}
\newpage	

\backmatter

\thispagestyle{empty}
\nocite{*}
\clearpage

\bibliographystyle{plain}

%\bibliography{literatura} 

%Prípadne môžete napísať literatúru priamo tu
\begin{thebibliography}{11}
	
\bibitem{srn} 
Jeffrey L.Elman
\textit{Finding Structure in Time}. 
University of California, San Diego, 1990
 
\bibitem{som} 
H. Ritter and T. Kohonen
\textit{Self-Organizing Semantic Maps} 
Helsinky University of Technology, 1982
  
\bibitem{recsom} 
Thomas Voegtlin
\textit{Recursive self-organizing maps}, 2002
 
\bibitem{msom} 
Marc Strickert, Barbara Hammer
\textit{Merge SOM for temporal data}
Technical University of Clausthal, 2005
\end{thebibliography}

%---koniec Referencii

% -------------------
%--- Prilohy---
% -------------------

%Nepovinná časť prílohy obsahuje materiály, ktoré neboli zaradené priamo  do textu. Každá príloha sa začína na novej strane.
%Zoznam príloh je súčasťou obsahu.
%
%\addcontentsline{toc}{chapter}{Appendix A}
%\input AppendixA.tex
%
%\addcontentsline{toc}{chapter}{Appendix B}
%\input AppendixB.tex

\end{document}






