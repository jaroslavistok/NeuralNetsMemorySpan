\chapter{Implementácia}

\section{Implementácia neurónových sietí}
Pre potreby merania pamäťovej hĺbky sme potrebovali 
veľmi modifikované implementácie neurónových sietí, 
preto sme sa rozhodli pre ich vlastnú implementáciu. 
Vlastná implementácia nám umožnila experimentovať s rôznymi typmi kontextov a rôznymi spôsobmi 
trénovania sietí, čo s existujúcimi implementáciami bolo veľmi nepraktické a v istých prípadoch nemožné.
Medzi špeciálne prípady patrí napríklad použitie modifikovaných kontextov, 
úprava excitačnej funkcie počas trénovania (zmenšovanie okolia), dynamické znižovanie rýchlosti
učenia siete počas trénovania/po jednotlivých epochách trénovania, vytvorenie pamäťového okna v jednotlivých neurónoch 
a samotné meranie pamäťovej hĺbky.

\section{Voľba programovacieho jazyka}
Ako implementačný jazyk sme zvolili Python pretože preň existuje veľké množstvo kvalitných knižníc pre prácu
s maticami a vektormi, či vykresľovanie grafov.
Python je veľmi populárny v oblasti strojového učenia. 
Ďaľšou výhodou je jednoduché spustenie skriptov
na linuxovom serveri, čo urýchľuje samotné trénovanie a hľadanie najoptimálnejších parametrov a umožnilo nám otestovať 
veľké množstvo kombinácii rôznych parametrov.
\subsection{Python}

Python je interpretovaný vysokoúrovňový programovací jazyk. 
Python kladie dôraz na jednoduchosť a čitateľnosť programov, ktoré sú v ňom naprogramované.
Je to jazyk, ktorý využíva dynamické typovanie a automatizovanú správu pamäte. Je to tiež multiplatformový 
jazyk a beží na všetkých bežne používaných platformách (Windows, Mac, Linux)

\section{Použíté knižnice}
Používame štandartný set knižníc pre implementáciu neurónových sietí: numpy, scipy.
Na vykresľovanie a vizualizáciu dát používame knižnice matplotlib a seaborn.
\subsection{Numpy}
Je knižnica, ktorá uľahčuje prácu s maticami, používaná je takmer všetkými existujúcimi
knižnicami, ktoré implementujú modely strojového učenia v Pythone. Má vysokú úroveň optimalizácie
a požíva veľmi optimalizované funkcie na prácu s maticami, ktoré sú naprogramované v jazyku C.
\subsection{Matplotlib}
Je knižnica na vykresľovanie grafov a vizualizáciu dát.
\subsection{Seaborn}
Je nadstavbou Matplotlib knižnice a zjednodušuje vykresľovanie rôznych grafov.

\section{Algoritmus hľadania najdlhšej spoločnej podpostupnosti viacerých reťazcov}
Na hľadanie najdlhšej spoločnej podpostupnosti viacerých reťazcov som použil relatívne jednoduchý naivný prístup. 
Z každej sekvencie sme si vytvorili všetky možné n-gramy (podpostupnosti), ktoré si ukladáme do množiny.
Pre každú sekvenciu v množine vytvoríme takúto množinu n-gramov. Potom spravíme prienik týchto množín a dĺžka najdlhšieho 
reťazca z tohto prieniku je dĺžka najdlhšej spoločnej podpostupnosti. Ide o naivný algoritmus, ktorý by bol pri
dlhých reťazcoch pamäťovo aj výpočtovo neoptimálny, avšak pre potreby merania pamäťovej hĺbky neurónových sietí
je postačujúci a dostatočne efektívny, keďže sekvencie v pamäťových oknách 
jednotlivých neurónov nie sú dlhé (do 30 znakov maximálne)
Použitie tohto algoritmu malo veľmi malý vplyv na rýchlosť trénovania.

\section{Reberov automat}
Na vytvorenie trénovacej množiny, ktorá pozotáva z reberových reťazcov sme si vytvorili vlastnú implementáciu pravdepodobnostného nedeterministického
konečného reberového automatu, pomocou ktorého generujeme reberové reťazce.
Každý stav okrem počiatočného a konečného stavu má práve dva prechody do ďaľšieho stavu, 
pričom v každom stave je 50\% šanca na prechod do jedného z možných stavov.
\begin{figure}[H]
    \centering
    \includegraphics[width=8cm]{assets/reber}
    \caption{Schéma reberovho automatu}
\end{figure}
