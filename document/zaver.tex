\chapter*{Záver}
V našej práci sme porovnali pamäťovú hĺbku niekoľkých typov neurónových sietí.
Hlavným cieľom bolo zistiť, či sa dajú rekurentné samoorganizujúce sa mapy použiť na vizualizáciu
sekvenčných vstupov, teda s akým dlhým kontextom do minulosti dokážu pracovať. Chceli sme tiež navrhnúť 
spôsob ako merať a hlavne porovnať pamäťovú hĺbku Elmanovej siete so rekurentnými SOM-kami, prípadne 
nájsť súvislosti.

V našich experimentoch sme porovnávali pamäťovú hĺbku dvoch základných typov rekurentných 
SOM - RecSOM a MSOM. Okrem toho sme si vytvorili z oboch modelov modifikované verzie - Activity RecSOM a Decaying MSOM.
Activity RecSOM umožňuje experimentovať s hodnotami aktivít neurónov, ktoré tvoria kontext siete. Upravili sme v nej
vzorec na výpočet aktivity, tak aby obsahoval meniteľný parameter, ktorý môžeme nastavovať na rôzne hodnoty.
To nám umožnilo preskúmať ďaľšie vlastnosti RecSOM a vplyv zloženia kontextu na hĺbku pamäte RecSOM.
Pri modifikácii MSOM sme sa rozhodli, že použijeme úplne odlišný kontext ako používa MSOM. Pri klasickej MSOM
je kontext tvorený lineárnou kombináciou vlastností víťazného neurónu z predchádzajúceho kroku, v našej modifikovanej 
verzii je kontext tvorený iba kombináciou minulých vstupov a teda nie je závislý od minulých stavov samotnej siete.

Zistili sme, že pri nízkodimenzionálnych vstupoch je vhodnejšie použiť RecSOM.


V experimente s Elmanovou sieťou sa nám podarilo nájsť spôsob ako vizualizovať súvislosti medzi 
aktiváciami neurónov na skrytej vrstve a posuvnými oknami na trénovacej množine.

% vysledky, ktore sme dosiahli
\section{Limity a nedostatky riešenia}
V našej práci sa nám nepodarilo nájsť spôsob ako zmerať a porovnať pamäťovú hĺbku Elmanovej siete.
Otázne je, či takáto sieť má nejakú kvantifikovateľnú pamäťovú hĺbku a či ju vieme odmerať a kvantifikovať.

\section{Možnosti ďaľšej práce}
\begin{itemize}
    \item Medzi ďaľšie možnosti patrí vyskúšanie ďaľších modifikácií rekurentných SOM. Napríklad pri Acitivty RecSOM 
    by sme mohli použiť euklidovskú vzdialenosť neurónu od víťazného neurónu a pozrieť sa na to aký vplyv by to malo 
    na pamäťovú hĺbku siete s rôznymi hodnotami parametra $\beta$.
    \item Experimentovať s ďaľšími kombináciami parametrov
    \item Vhodné by bolo lepšie kvantifikovať pamäťovú hĺbku v prípade, že neuróny rekurentných SOM majú vo svojom
    receptívnom poli uloženú iba jednu sekvenciu. Tiež by bolo vhodné analyzovať lepšie súvislosti medzi kvantizačnou chybou a hodnotami pamäťovej hĺbky.
    \item Spraviť podrobnú matematickú analýzu experimentov
    \item Hľadanie a skúšanie alternatívnych spôsobov ako zmerať pamäťovú hĺbku Elmanovej siete. Napríklad podrobnejšiou
    analýzou dendrogramov.
\end{itemize}





