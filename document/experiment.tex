% podobne prace?

\chapter{Experiment}
Experimentom v našej práci je meranie hĺbky pamäte a vyhodnotenie vplyvu rôznych
parametrov a typov kontextov na hĺbku pamäte. Súčasťou nášho experimentu je aj nájdenie 
optimálnej kombinácie parametrov pre všetky typy porovnávaných sietí.

\section{Hľadanie optimálnych parametrov sietí}
Na to aby sme mohli porovnať hĺbku pamäte rôznych sietí sme museli nájsť kombináciu parametrov
pri ktorých daný typ siete dosahujú najlepšie výsledky. 
Pri trénovaní SOM môžeme meniť a optimalizovať veľké množstvo parametrov. 
Experimentami sme zistili, že na hĺbku pamäte siete majú vplyv iba niektoré z nich. 
Najdôležitejšie parametre, ktoré vplývajú na hĺbku pamäte neurónovej siete sú parametre $\alpha$ a $\beta$
vo vzťahu pre výpočet vzdialenosti vstupného vektora od určitého neurónu v siete (čiže od jeho váhového a kontextového vektora).
% TODO pridat rovnicu na ilustraciu
Tieto dva parametre určujú pomer dôležitosti aktuálneho vstupu a dôležitosť kontextu, pri výpočte vzdialenosti (kvantizačnej chyby).

Experiment prebiehal nasledujúcim spôsobom:
\begin{itemize}
    \item Vybrali sme vhodnú trénovaciu sekvenciu, počet epôch trénovania a dostatočnú veľkosť pamäťového okna
    \item Spustili sme trénovanie na všetkých kombináciach týchto dvoch parametrov s krokom 0.1
    \item Hodnoty pamäťovej hĺbky sme ukladali do súboru
    \item Na záver sme vykreslili heatmapu, ktorá znázorňuje aká bola pamäťová hĺbka pre rôzne kombinácie parametrov.
\end{itemize}

% TODO vyber rychlosti ucenia (konstantna / postupne zmensujuca?)

Počet epôch sme určili na základe kvantizačnej chyby.
Počet epôch sme postupne zvyšovali a keď kvantizačná chyba prestala signifikantne klesať, resp. dosiahla 
svoje minimum zastavali sme ho na tejto hodnote a ďalej nezvyšovali. SOM sa dokážu relatívne rýchlo učiť a 
teda počet epôch nemusí byť vysoký, čo je veľkou výhodou pri experimentovaní, kedže trénovanie netrvá príliš dlhú dobu
a tým pádom sme mohli vyskúšať viac kombinácii a modifikácii.

Dostatočnú veľkosť pamäťového okna sme určili podobne ako počet epôch. Parameter sme postupne zvyšovali
a zastavili na hodnote, keď pamäťová hĺbka siete prestala stúpať, čiže veľkosť pamäťového okna už
neovplyvňovala hĺbku pamäte a ďalšie zvyšovanie parametra nemalo zmysel. 

Na základe tohto sme zistili, že pre každý typ siete sú ideálne hodnoty týchto parametrov odlišné.


\section{Experiment so SRN a Reberovým automatom}



